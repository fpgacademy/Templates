\documentclass[11pt, twoside, pdftex]{article}

\setlength\topmargin{-0.25in}
\setlength\headheight{0in}
\setlength\headsep{0.35in}
\setlength\textheight{8.5in}
\setlength\textwidth{7in}
\setlength\oddsidemargin{-0.25in}
\setlength\evensidemargin{-0.25in}
\setlength\parindent{0.25in}
\setlength\parskip{0in} 

\pdfpagewidth 8.5in
\pdfpageheight 11in

% listings is a package that supports encapsulating source code in LaTeX conveniently

\usepackage{listings}
% add support for graphics
\usepackage{graphicx}
\usepackage[usenames, dvipsnames]{color}

\def\expandparam\lstinputlisting[#1]#2{\edef\tmp{\noexpand\lstinputlisting[#1]{#2}}\tmp}

\widowpenalty 10000
\clubpenalty 10000

%%%%%%%%%%%%%%%%%%%% Source Code Formatting %%%%%%%%%%%%%%%%%%%%
\definecolor{globalCommentColour}{rgb}{0.588,0.588,0.588}

%%%%%%%%%%%%%%%%%%%%%%%%%%%%%%%%%%%%%%%%%%%%%%%%%%%%
% Defining a NiosII ASM highlighter for lstlisting
\lstdefinelanguage[NiosII]{Assembler} {
 	morekeywords={add, addi, and, andhi, andi, beq, bge, bgeu, bgt, bgtu, ble,  bleu, blt, bltu, bne, br, break,% 
 	bret, call, callr, cmpeq, cmpeqi, cmpge, cmpgei, cmpgeu, cmpgeui, cmpgt, cmpgti, cmpgtu, cmpgtui, cmple,%
 	cmplei, cmpleu, cmpleui, cmplt, cmplti, cmpltu, cmpltui, cmpne, cmpnei, custom, div, divu, eret, flushd,%
 	flushda, flushi, flushp, initd, initda, initi, jmp, jmpi, ldb, ldbio, ldbu, ldbuio, ldh, ldhio, ldhu, ldhuio,%
 	ldw, ldwio, mov, movhi, movi, movia, movui, mul, muli, mulxss, mulxsu, mulxuu, nextpc, nop, nor, or, orhi, ori,%
 	rdctl, rdprs, ret, rol, roli, ror, sll, slli, sra, srai, srl, srli, stb, stbio, sth, sthio, stw, stwio,%
 	sub, subi, sync, trap, wrctl, wrtcl, wrprs, xor, xori, xorhi, xori},% 	
 	morekeywords=[2]{.abort, .ABORT, .align, .app-file, .ascii, .asciz, .balign, .byte, .comm, .data, .def,%
 	.desc, .dim, .double, .eject, .else, .end, .endef, .endif, .equ, .equiv, .err, .extern, .file, .fill, .float,%
 	.global, .globl, .hword, .ident, .if, .include, .int, .irp, .irpc, .lcomm, .lflags, .line, .linkonce, .ln,%
 	.list, .long, .macro, .mri, .nolist, .octa, .org, .p2align, .psize, .quad, .rept, .sbttl, .scl, .section,%
 	.set, .short, .single, .size, .sleb128, .skip, .space, .stadb, .stabn, .stabs, .string, .symver, .tag,%
 	.text, .title, .type, .val, .uleb128, .word},% 	
 	morekeywords=[3]{et, bt, gp, sp, fp, ea, sstatus, ra, pc, status, estatus, bstatus, ienable, ipending, cpuid,%
 	exception, pteaddr, tlbacc, tlbmisc, eccinj, badaddr, config, mpubase, mpuacc},% 	
 	sensitive=t,%
 	alsoletter=.,%
	morestring=[b]",%
 	morecomment=[s]{/*}{*/},%
 	morecomment=[l]\#,%
   }[keywords,comments,strings]
   
   %% NOTE: morekeywords=[2] are GNU directives.
   
   \definecolor{niosInstructionColour}{rgb}{0.000,0.608,0.000}
   \definecolor{niosDirectiveColour}{rgb}{0.000,0.000,0.902}
   \definecolor{niosSpecialRegColour}{rgb}{0.000,0.000,0.000}
   \definecolor{niosStringColour}{rgb}{0.808,0.482,0.000}
   
   %% NOTE: To make bold use: =\bfseries\color{<colour>}
   \lstdefinestyle{defaultNiosStyle} {
   language=[NiosII]{Assembler},
   stringstyle=\color{niosStringColour},
   keywordstyle=\color{niosInstructionColour},
   keywordstyle=[2]\color{niosDirectiveColour},
   keywordstyle=[3]\itshape\color{niosSpecialRegColour}
   }
%%%%%%%%%%%%%%%%%%%%%%%%%%%%%%%%%%%%%%%%%%%%%%%%%%%%

%%%%%%%%%%%%%%%%%%%%%%%%%%%%%%%%%%%%%%%%%%%%%%%%%%%%
% Defining a Nios V ASM highlighter for lstlisting
\lstdefinelanguage[NiosV]{Assembler} {
 	morekeywords={la, li, lb, lh, lw, lbu, lhu, sb, sh, sw, sll, slli, srl, srli, sra, srai,
    add, addi, sub, lui, auipc, xor, xori, or, ori, and, andi, not, slt, slti, sltu, sltiu,
    seqz, snez, sgtz, sgt, sgtu, slt, mul, mulh, mulhsu, mulhu, div, divu, rem, remu,
    ret, mret, beqz, bnez, bgtz, bltz, blez, bgt, bgtu, ble, bleu,
    beq, bne, blt, bge, bgez, bltu, bgeu, j, jr, jal, jalr, mv, neg, ret, scall, break, nop
    csrrw, csrrwi, csrrs, csrrsi, csrrc, csrrci, csrr, csrw, csrwi, csrs, csrsi, csrc,
    csrci, ecall, wfi},% 	
 	morekeywords=[2]{.align, .ascii, .asciiz, .byte, .data, .double, .equ, .extern, .include,
    .float, .global, .hword, .kdata, .ktext, .set, .skip, .end, .space, .text, .word},% 	
 	morekeywords=[3]{zero, ra, sp, gp, tp, s0, fp, t0, t1, t2, t3, t4, t5, t6,
    s1, s2, s3, s4, s5, s6, s7, s8, s9, s10, s11, a0, a1, a2, a3, a4, a5, a6, a7,
    ft0, ft1, ft2, ft3, ft4, ft5, ft6, ft7, fs0, fs1, fs2, fs3, fs4, fs5, fs6, fs7, 
    fs8, fs9, fs10, fs11, fa0, fa1, fa2, fa3, fa4, fa5, fa6, fa7},% 	
 	sensitive=t,%
 	alsoletter=.,%
	morestring=[b]",%
 	morecomment=[s]{/*}{*/},%
 	morecomment=[l]\#,%
   }[keywords,comments,strings]
   
   %% NOTE: morekeywords=[2] are GNU directives.
   
   \definecolor{niosVInstructionColour}{rgb}{0.000,0.608,0.000}
   \definecolor{niosVDirectiveColour}{rgb}{0.000,0.000,0.902}
   \definecolor{niosVSpecialRegColour}{rgb}{0.000,0.000,0.000}
   \definecolor{niosVStringColour}{rgb}{0.808,0.482,0.000}
   \definecolor{NiosVCommentColour}{rgb}{0.000,0.502,0.000}
   \definecolor{brown}{rgb}{.59,.29,.0}
   
   %% NOTE: To make bold use: =\bfseries\color{<colour>}
   \lstdefinestyle{defaultNiosVStyle} {
   language=[NiosV]{Assembler},
   stringstyle=\color{niosVStringColour},
   keywordstyle=\color{niosVInstructionColour},
   keywordstyle=[2]\color{niosVDirectiveColour},
   keywordstyle=[3]\itshape\color{niosVSpecialRegColour},
   commentstyle=\small\color{brown}\ttfamily
   }
%%%%%%%%%%%%%%%%%%%%%%%%%%%%%%%%%%%%%%%%%%%%%%%%%%%%

%%%%%%%%%%%%%%%%%%%%%%%%%%%%%%%%%%%%%%%%%%%%%%%%%%%%
% Defining a ArmA9 ASM highlighter for lstlisting
\lstdefinelanguage[ArmA9]{Assembler} {
 	morekeywords={ADC, ADD, ADDS, AND, ANDS, B, BAL, BEQ, BGE, BGT, BL, BLT, BIC, BKPT, BLX, BNE, BX, CDP, CLZ, CMN, CMP, EOR,%
 	EORS, LDC, LDM, LDR, LDRB, LDRBT, LDRH, LDRSB, LDRSH, LDRT, LSL, MCR, MLA, MOV, MOVW, MOVT, MRC, MRS, MSR, MUL, MVN, ORR, PLD,%
 	ROR, RSB, RSC, SBC, SMLAL, SMULL, STC, STM, STR, STRB, STRBT, STRH, STRT, SUB, SUBS, SWI, SWP, SWPB, TEQ, UMLAL,
 	PUSH, POP, MOVS, RORS, LSR},%
 	morekeywords=[2]{.abort, .ABORT, .align, .app-file, .ascii, .asciz, .balign, .byte, .comm, .data, .def,%
 	.desc, .dim, .double, .eject, .else, .end, .endef, .endif, .equ, .equiv, .err, .extern, .file, .fill, .float,%
 	.global, .globl, .hword, .ident, .if, .include, .int, .irp, .irpc, .lcomm, .lflags, .line, .linkonce, .ln,%
 	.list, .long, .macro, .mri, .nolist, .octa, .org, .p2align, .psize, .quad, .rept, .sbttl, .scl, .section,%
 	.set, .short, .single, .size, .sleb128, .skip, .space, .stadb, .stabn, .stabs, .string, .symver, .tag,%
 	.text, .title, .type, .val, .vectors, .uleb128, .word},%
 	morekeywords=[3]{SP, PC, MIDR, CTR, TCMTR, TLBTR, MPIDR, ID_PFR0, ID_PFR1, ID_DFR0, ID_MMFR0, ID_MMFR1, ID_MMFR2,%
 	ID_MMFR3, ID_ISAR0, ID_ISAR1, ID_ISAR2, ID_ISAR3, ID_ISAR4, CCSIDR, CLIDR, AIDR, CSSELR, TTBR0, TTRB1, TTBR2, DACR,%
 	DFSR, IFSR, ADFSR, AIFSR, DFAAR, IFAR, ICIALLUIS, BPIALLIS, PAR, ICIALLU, ICIMVAU, BPIALL, DCIMVAC, DCISW, V2PCWPR,%
 	DCCVAC, DCCSW, DDIMVAC, DCISW, TLBALLIS, TLBIMVAIS, TLBIASIDIS, TLBIMVAAIS, TLBIALL, TLBIMVA, TLBIASID, TLBIMVAA,%
 	PMCR, PMCNTENSET, PMCNTENCLR, PMOVSR, PMSWINC, PMSELR, PMXEVTYPER, PMXEVCNTR, PMUSERENR, PMINTENSET, PMINTENCLR,%
 	PRRR, NRRR, PLEIDR, PLEASR, PLEFSR, PLEUAR, PLEPCR, VBAR, MVBAR, ISR, FCSEIDR, CONTEXTIDR, TPIDRURW, TPIDRURO, TPIDRPRW},%
 	sensitive=f,%
 	alsoletter=.,%
	morestring=[b]",%
 	morecomment=[s]{/*}{*/},%
 	morecomment=[l]{//},%
   }[keywords,comments,strings]
   
   %% NOTE: morekeywords=[2] are GNU directives.
   
   \definecolor{armInstructionColour}{rgb}{0.000,0.608,0.000}
   \definecolor{armDirectiveColour}{rgb}{0.000,0.000,0.902}
   \definecolor{armSpecialRegColour}{rgb}{0.000,0.000,0.000}
   \definecolor{armStringColour}{rgb}{0.808,0.482,0.000}
   
   \lstdefinestyle{defaultArmStyle} {
   language=[ArmA9]{Assembler},
   stringstyle=\color{armStringColour},
   keywordstyle=\color{armInstructionColour},
   keywordstyle=[2]\color{armDirectiveColour},
   keywordstyle=[3]\itshape\color{armSpecialRegColour}
   }
%%%%%%%%%%%%%%%%%%%%%%%%%%%%%%%%%%%%%%%%%%%%%%%%%%%%

%%%%%%%%%%%%%%%%%%%%%%%%%%%%%%%%%%%%%%%%%%%%%%%%%%%%
% Defining style for the verilog.

\definecolor{verilogCommentColour}{rgb}{0.000,0.502,0.000}

\lstdefinestyle{defaultVerilogStyle} {
language={Verilog},
keywordstyle=\color{blue},
commentstyle=\color{verilogCommentColour}
}
%%%%%%%%%%%%%%%%%%%%%%%%%%%%%%%%%%%%%%%%%%%%%%%%%%%%

%%%%%%%%%%%%%%%%%%%%%%%%%%%%%%%%%%%%%%%%%%%%%%%%%%%%
% Defining style for the vhdl.
\lstdefinestyle{defaultVHDLStyle} {
language={VHDL},
keywordstyle=\color{blue},
commentstyle=\color{verilogCommentColour}
}
%%%%%%%%%%%%%%%%%%%%%%%%%%%%%%%%%%%%%%%%%%%%%%%%%%%%

%%%%%%%%%%%%%%%%%%%%%%%%%%%%%%%%%%%%%%%%%%%%%%%%%%%%
% Java
\definecolor{javaStringColour}{rgb}{0.808,0.482,0}
%%%%%%%%%%%%%%%%%%%%%%%%%%%%%%%%%%%%%%%%%%%%%%%%%%%%

%%%%%%%%%%%%%%%%%%%%%%%%%%%%%%%%%%%%%%%%%%%%%%%%%%%%
% Defining language styles
% C
\definecolor{CStringColour}{rgb}{0.808,0.482,0}
%%%%%%%%%%%%%%%%%%%%%%%%%%%%%%%%%%%%%%%%%%%%%%%%%%%%

%%%%%%%%%%%%%%%%%%%%%%%%%%%%%%%%%%%%%%%%%%%%%%%%%%%%
% Defining extended LaTeX language.
\lstdefinelanguage[LocalLaTeX]{TeX}[LaTeX]{TeX}%
 	{moretexcs={bf, it, sf, lstset},%
   	}%

\lstdefinestyle{defaultLocalLatexStyle} {
language=[LocalLatex]{TeX},
keywordstyle=\color{blue}\bfseries,
keywordstyle=[2]\color{blue},
keywordstyle=[3]\color{blue}\bfseries
}
%%%%%%%%%%%%%%%%%%%%%%%%%%%%%%%%%%%%%%%%%%%%%%%%%%%%

\lstset{
%language = C,
%language = Verilog,
%basicstyle=\color{black}\rmfamily\ttfamily,
basicstyle=\small\color{black}\ttfamily,
commentstyle=\small\color{globalCommentColour}\itshape\ttfamily,
keywordstyle=\small\color{blue}\bfseries\ttfamily,
showstringspaces=false,
frame=none, %lines % boxed listings
breaklines=true,
breakatwhitespace=true,
tabsize=4
}
%%%%%%%%%%%%%%%%%%%%%%%%%%%%%%%%%%%%%%%%%%%%%%%%%%%%%%%%%%%%%%%%


%\usepackage[centering]{geometry}.
%%%%%%%%%%%%%%%%%%%%%%%%%%%%%%%%%%%%%%%%%%%%%%%%%%%
% Document Settings
\usepackage[labelsep=period]{caption}
% we can choose a better font later
%\usepackage{palatino}
\usepackage{fourier}
%\fontencoding{T1}
% include common used symbols
\usepackage{textcomp}
% add support for graphics
\usepackage{graphicx}
\usepackage[usenames, dvipsnames]{color}
% enable to draw thick or thin table hlines
\setlength{\doublerulesep}{\arrayrulewidth}
\usepackage{longtable}
\setlongtables
%\usepackage{array}
% It may be better to use PDFLaTeX as it can generate bookmarks for the
% document

% Add some useful packages
\usepackage{ae,aecompl}
\usepackage{epsfig,float,times}

% reset the font for section
\usepackage{sectsty}
%\allsectionsfont{\fontfamily{ptm}\selectfont}
\allsectionsfont{\usefont{OT1}{phv}{bc}{n}\selectfont}

% use compact space for sections
\usepackage[compact]{titlesec}
\titlespacing{\section}{0pt}{0.2in}{*0}
\titlespacing{\subsection}{0pt}{0.1in}{*0}
\titlespacing{\subsubsection}{0pt}{0.05in}{*0}

% fancyhdr header and footer customization
\usepackage{layout}
\usepackage{fancyhdr}
\pagestyle{fancy}
\fancyhead{}
\fancyhead[R]{\textit{\tiny{\textBar}}}
\fancyfoot{}
\fancyfoot[LO,
RE]{\textrm{\href{https://www.fpgacademy.org}{\small \longteamname}} \\ {\small \datePublished }}
\fancyfoot[RO, LE]{\small \thepage}
% two-side settings
%\fancyhead{} % clear all header fields
%\fancyfoot{} % clear all footer fields
%\fancyfoot[LE,RO]{\thepage}
\renewcommand{\headrulewidth}{2pt}
\renewcommand{\headrule}{{\color{blue} \hrule width\headwidth height\headrulewidth \vskip-\headrulewidth}}
\renewcommand{\footrulewidth}{0pt}

% Format the footer on page 1
\fancypagestyle{plain}{
\fancyhead{}
\fancyfoot{}
\fancyfoot[LO,
RE]{\textrm{\href{https://www.fpgacademy.org}{\small \longteamname}} \\ {\small \datePublished }}
\fancyfoot[RO, LE]{\small \thepage}
\renewcommand{\headrulewidth}{0pt}
}
% adjust some setting to try to make the figure stay in the same page with text
% Reference: 	http://www.cs.uu.nl/~piet/floats/node1.html
%   			http://mintaka.sdsu.edu/GF/bibliog/latex/floats.html
%   General parameters, for ALL pages:
\renewcommand{\topfraction}{0.9}	% max fraction of floats at top
\renewcommand{\bottomfraction}{0.8}	% max fraction of floats at bottom
%   Parameters for TEXT pages (not float pages):
\setcounter{topnumber}{3}
\setcounter{bottomnumber}{3}
\setcounter{totalnumber}{5}     % 2 may work better
\setcounter{dbltopnumber}{2}    % for 2-column pages
\renewcommand{\dbltopfraction}{0.9}	% fit big float above 2-col. text
\renewcommand{\textfraction}{0.07}	% allow minimal text w. figs
%   Parameters for FLOAT pages (not text pages):
\renewcommand{\floatpagefraction}{0.7}	% require fuller float pages
% N.B.: floatpagefraction MUST be less than topfraction !!
\renewcommand{\dblfloatpagefraction}{0.7}	% require fuller float pages

%%%%%%%%%%%%%%%%%%%%%%%%%
% Add title
\title{\fontfamily{phv}\selectfont{\doctitle} }
\chead{ \small{\textsc{\bfseries \dochead} } }
%%%%%%%%%%%%%%%%%%%%%%%%%

% set no indent for paragraph
\setlength{\parindent}{0em}
\addtolength{\parskip}{11pt}
\newcommand{\compact}{[topsep=0pt]}
% use this package to reduce space
\usepackage{enumitem}
\usepackage{multirow}
\usepackage{rotating}
\usepackage{pifont}
\usepackage{dingbat}
\newcommand{\itemsecond}{$\circ$}
%
%%%%%%%%%%%%%%%%%%
\date{}
\author{}
%%%%%%%%%%%%%%%%%%
\newcommand{\de}{DE-series}
\newcommand{\up}{FPGAcademy}
\newcommand{\fabric}{Avalon Switch Fabric}
\newcommand{\red}[1]{{\color{red}\sf{#1}}}
\newcommand{\TODO}[1]{\textcolor{red}{\textbf{TODO}: #1}}
\def\registered{{\ooalign{\hfil\raise .00ex\hbox{\scriptsize R}\hfil\crcr\mathhexbox20D}}}

% enable url and reference(bookmarks) in pdf
\usepackage{url}
\usepackage[pdftex, colorlinks]{hyperref}
\hypersetup{%
pdftitle={\PDFTitle},
linkcolor=blue,
hyperindex=true,
pdfauthor={\longteamname},
pdfkeywords={FPGAcademy, Academic Program, Example System},
bookmarksnumbered,
bookmarksopen=false,
filecolor=blue,
pdfstartview={FitH},
urlcolor=blue,
plainpages=false,
pdfpagelabels=true,
linkbordercolor={1 1 1} %no color for link border
}%
%%%%%%%%%%%%%%%%%%%%%%%%%%%%%%%%%%%%%%%%%%%%%%%%%%%
\setlength{\fboxsep}{0.7pt}
\setlength{\fboxrule}{0.5pt}

\raggedbottom
\widowpenalty 10000
\clubpenalty 10000

